%-------------------------------------------------------------------------------
% Introduction.tex
%-------------------------------------------------------------------------------
% 2024-03-29
% curtis
%-------------------------------------------------------------------------------
% Description of LaTeX project
%-------------------------------------------------------------------------------

\makeatletter
\def\input@path{{../}}
\makeatother
\documentclass[../TacSubGameRules.tex]{subfiles}

\begin{document}

\section*{Introduction}

\gametitle~ models some basic elements of an anti-surface warfare (ASuW) engagement, an important mission in undersea warfare (USW) as ultimately most things of interest to humans float on the surface of the sea.
In this solitaire game, the player commands a nuclear fast-attack submarine (SSN) and stalks a surface action group escorting a landing helicopter dock (LHD) containing marines about to engage in nefarious activities in a land where they do not belong.
The LHD is a high-value unit (HVU) that the SSN must sink.

This engagement consists of two major phases: the stalking phase, and the disengagement phase.
During the stalking phase, the player approaches the LHD until the SSN is in a desirable attack position.
The player must manage speed, angle of approach, depth, and position, as the player approaches the target.
Each of these factors influences how detectable the SSN is and the likelihood of a successful attack.
Being detected too soon may make sinking the LHD difficult and survival unlikely.

During the second phase of the game, the player must disengage from the surface units, who now pursue the submarine to sink it.
In addition to pursuing frigates, helicopters (helos) armed with torpedoes will start searching the area for the SSN using their dipping sonar.
Should they find the SSN, they will loose their torpedoes and perhaps sink the fleeing SSN.
If the SSN escapes the area, the player wins.

\gametitle~ was originally designed as a postcard game with a terse ruleset.
These rules should be in agreement with the original postcard rules; however, they state how to play more plainly, including elaborating what was once implied by the postcard rules, provide examples, and even give tips for playing the game.
However, the original postcard rules may still be useful as a pocket rules reference; again, the two sets of rules should be in complete agreement.
To help players find rules, the rules are cross-referenced using a notation resembling ``3.23'', in this case referring to rule 23 in section 3.

Players already familiar with USW may notice aspects of the game that are not true to real life.
When I started designing this game, I decided not to ever ask questions about important parameters of ASuW engagements, including ranges, time frames, the rate at which something happens, and so on.
I did not even look at Wikipedia.
As I planned to post the game to Twitter without asking anyone's permission, I thought this restriction would be important to ensuring I did not say something I should not say.
However, the result was some interactions resulting in unrealistic situations.
I beg your forgiveness, and if someone wants to have me design a game that's more realistic, they should consider paying me for the trouble.
This game is partly a prototype for what tactical submarine games that use more realistic operational parameters might look like.

\begin{wrapfigure}{r}{20mm}
    \centering
    \includegraphics[width=20mm]{vidlink}
\end{wrapfigure}
I created a video teaching how to play \gametitle~ using a Vassal module I created.
In addition to being a teach-and-play (using the postcard rules), I also provide comments on the game design.
I am pleased with the video and am glad that the unrehearsed scenario turned out to be interesting.
In addition to the code in the picture, here is a URL to the video: \url{https://www.youtube.com/watch?v=Wq7-l8As7Zc}.

\end{document}

